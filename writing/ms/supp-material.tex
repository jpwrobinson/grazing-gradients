\documentclass[12pt,]{article}
\usepackage{lmodern}
\usepackage{amssymb,amsmath}
\usepackage{ifxetex,ifluatex}
\usepackage{fixltx2e} % provides \textsubscript
\ifnum 0\ifxetex 1\fi\ifluatex 1\fi=0 % if pdftex
  \usepackage[T1]{fontenc}
  \usepackage[utf8]{inputenc}
\else % if luatex or xelatex
  \ifxetex
    \usepackage{mathspec}
  \else
    \usepackage{fontspec}
  \fi
  \defaultfontfeatures{Ligatures=TeX,Scale=MatchLowercase}
\fi
% use upquote if available, for straight quotes in verbatim environments
\IfFileExists{upquote.sty}{\usepackage{upquote}}{}
% use microtype if available
\IfFileExists{microtype.sty}{%
\usepackage{microtype}
\UseMicrotypeSet[protrusion]{basicmath} % disable protrusion for tt fonts
}{}
\usepackage[margin=1in]{geometry}
\usepackage{hyperref}
\hypersetup{unicode=true,
            pdfborder={0 0 0},
            breaklinks=true}
\urlstyle{same}  % don't use monospace font for urls
\usepackage{graphicx,grffile}
\makeatletter
\def\maxwidth{\ifdim\Gin@nat@width>\linewidth\linewidth\else\Gin@nat@width\fi}
\def\maxheight{\ifdim\Gin@nat@height>\textheight\textheight\else\Gin@nat@height\fi}
\makeatother
% Scale images if necessary, so that they will not overflow the page
% margins by default, and it is still possible to overwrite the defaults
% using explicit options in \includegraphics[width, height, ...]{}
\setkeys{Gin}{width=\maxwidth,height=\maxheight,keepaspectratio}
\IfFileExists{parskip.sty}{%
\usepackage{parskip}
}{% else
\setlength{\parindent}{0pt}
\setlength{\parskip}{6pt plus 2pt minus 1pt}
}
\setlength{\emergencystretch}{3em}  % prevent overfull lines
\providecommand{\tightlist}{%
  \setlength{\itemsep}{0pt}\setlength{\parskip}{0pt}}
\setcounter{secnumdepth}{0}
% Redefines (sub)paragraphs to behave more like sections
\ifx\paragraph\undefined\else
\let\oldparagraph\paragraph
\renewcommand{\paragraph}[1]{\oldparagraph{#1}\mbox{}}
\fi
\ifx\subparagraph\undefined\else
\let\oldsubparagraph\subparagraph
\renewcommand{\subparagraph}[1]{\oldsubparagraph{#1}\mbox{}}
\fi

%%% Use protect on footnotes to avoid problems with footnotes in titles
\let\rmarkdownfootnote\footnote%
\def\footnote{\protect\rmarkdownfootnote}

%%% Change title format to be more compact
\usepackage{titling}

% Create subtitle command for use in maketitle
\providecommand{\subtitle}[1]{
  \posttitle{
    \begin{center}\large#1\end{center}
    }
}

\setlength{\droptitle}{-2em}

  \title{}
    \pretitle{\vspace{\droptitle}}
  \posttitle{}
    \author{}
    \preauthor{}\postauthor{}
    \date{}
    \predate{}\postdate{}
  
\usepackage{pdflscape}
\newcommand{\blandscape}{\begin{landscape}}
\newcommand{\elandscape}{\end{landscape}}

\begin{document}

\hypertarget{habitat-and-fishing-control-grazing-potential-on-coral-reefs}{%
\subsection{Habitat and fishing control grazing potential on coral
reefs}\label{habitat-and-fishing-control-grazing-potential-on-coral-reefs}}

James PW Robinson, Jamie M McDevitt-Irwin, Jan-Claas Dajka, Jeneen
Hadj-Hammou, Samantha Howlett, Alexia Graba-Landry, Andrew S Hoey,
Kirsty L Nash, Shaun K Wilson, Nicholas AJ Graham

\hypertarget{supplementary-methods}{%
\paragraph{Supplementary Methods}\label{supplementary-methods}}

\emph{Region details}

In Seychelles, 21 reefs were surveyed in 2008, 2014, and 2017 on two
inhabited islands (Mahe, Praslin). Surveys were conducted on the reef
slope at 9-12 m depth, and stratified to include carbonate fringing
reefs, granitic rocky reefs with coral growth, and patch reef habitats
on a sand, rubble, or rock base (Fig. S1B). Surveys were repeated for
either 8 (2008) or 16 (2011, 2014, 2017) replicates at each reef, which
were located at least 15 m away from each other. To ensure that survey
effort was comparable among Seychelles reefs, we only considered surveys
from the first 8 replicates (per site per survey year). Overall, the
surveys covered up to 0.5 km of reef front and 2,500
m\textsuperscript{2} of reef habitat, including 672 point counts over 4
surveyed years. Reefs were categorised by their exploitation status,
with 9 sites in small protected areas and 12 sites supporting artisanal
fisheries.

In the Chagos archipelago, 25 reefs were surveyed on four uninhabited
atolls in 2010 (Fig. S1B). Surveys were stratified to include sheltered
(9) and exposed (9) habitats, and four replicate transects were
conducted at each site, resulting in 100 total transects. All reefs were
categorised as remote.

In Maldives, 11 reefs were surveyed on one atoll (Huvadhoo) in 2013
(Fig. S1B). Surveys were conducted on the reef slope for 4 replicates
per reef, resulting in 44 total transects. All reefs were categorised as
fished.

In Australia, five reefs were surveyed on the central Great Barrier Reef
in 2010 and 2011 (Wheeler, Davies, Rib, Trunk, John Brewer) (Graham et
al.~2014) (Fig. S1B). Reefs were stratified to include 3 wave exposed
and 3 wave sheltered locations (6 per reef), which were further divided
into reef slope (7-9 m depth), reef crest (2-3 m depth), and reef flat
(100 m distance from crest). Each location and habitat type was surveyed
with four replicate transects. We used data for surveys conducted on the
reef slope, which produced a dataset of 24 transects per reef and 120
transects in total. Davies, Rib, Trunk and John Brewer were categorised
as fished, and Wheeler was categorised as protected (no-take zone).

\emph{Benthic categories}

Across all reefs, we detected four benthic regimes characterised by 1)
hard coral dominance, 2) macroalgal dominance, 3) high availability of
bare substrate, and 4) rubble reefs (Fig. S1B). Coral dominance was the
most common regime, detected at 41 reefs across all four regions,
whereas bare substrate regimes were only present in Seychelles (9) and
Chagos (6). Macroalgal dominance was detected on five Seychelles reefs
and nine GBR reefs, while rubble reefs were only present in Seychelles
(6 reefs).

~

\begin{center}\includegraphics[width=480px]{../../figures/Figure1} \end{center}

\textbf{Figure S1 \textbar{} Map of study sites with benthic habitat
regimes (B) and herbivore biomass levels (C).} Survey sites are coloured
by regimes identified in k-cluster analysis (rubble = yellow, macroalgae
= green, substrate = blue, coral = red), and bar plots show mean grazing
biomass (± 2 standard errors) for croppers and scrapers.

\newpage

\begin{center}\includegraphics[width=480px]{../../figures/FigureS1_scrape_size} \end{center}

\textbf{Figure S2 \textbar{} Size effects on scraper bite rates (A) and
bite area (B).} Lines indicate median posterior predictions with 95\%
certainty intervals, excluding species and genera effects, across the
range of observed body sizes (total length, cm). Points are observed
bite rates or bite areas coloured by genera.

\newpage

\begin{center}\includegraphics[width=480px]{../../figures/FigureS2_cropper_bites} \end{center}

\textbf{Figure S3 \textbar{} Cropper bite rate predictions (A) and
observed cropper function in UVC (B)} Predicted bite rates are median
posterior predictions with 95\% certainty intervals (A), and boxplots
are site-level observed cropping function for each reef, coloured by UVC
region.

\newpage

\begin{center}\includegraphics[width=480px]{../../figures/FigureS3_scraper_bites} \end{center}

\textbf{Figure S4 \textbar{} Scraper bite rate predictions (A) and
observed scraping function in UVC (B)} Predicted bite rates are median
posterior predictions with 95\% certainty intervals (A), and boxplots
are site-level observed scraping function for each reef, coloured by UVC
region.

\newpage


\end{document}

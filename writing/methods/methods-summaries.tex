\documentclass[]{article}
\usepackage{lmodern}
\usepackage{amssymb,amsmath}
\usepackage{ifxetex,ifluatex}
\usepackage{fixltx2e} % provides \textsubscript
\ifnum 0\ifxetex 1\fi\ifluatex 1\fi=0 % if pdftex
  \usepackage[T1]{fontenc}
  \usepackage[utf8]{inputenc}
\else % if luatex or xelatex
  \ifxetex
    \usepackage{mathspec}
  \else
    \usepackage{fontspec}
  \fi
  \defaultfontfeatures{Ligatures=TeX,Scale=MatchLowercase}
\fi
% use upquote if available, for straight quotes in verbatim environments
\IfFileExists{upquote.sty}{\usepackage{upquote}}{}
% use microtype if available
\IfFileExists{microtype.sty}{%
\usepackage{microtype}
\UseMicrotypeSet[protrusion]{basicmath} % disable protrusion for tt fonts
}{}
\usepackage[margin=1in]{geometry}
\usepackage{hyperref}
\hypersetup{unicode=true,
            pdfborder={0 0 0},
            breaklinks=true}
\urlstyle{same}  % don't use monospace font for urls
\usepackage{graphicx,grffile}
\makeatletter
\def\maxwidth{\ifdim\Gin@nat@width>\linewidth\linewidth\else\Gin@nat@width\fi}
\def\maxheight{\ifdim\Gin@nat@height>\textheight\textheight\else\Gin@nat@height\fi}
\makeatother
% Scale images if necessary, so that they will not overflow the page
% margins by default, and it is still possible to overwrite the defaults
% using explicit options in \includegraphics[width, height, ...]{}
\setkeys{Gin}{width=\maxwidth,height=\maxheight,keepaspectratio}
\IfFileExists{parskip.sty}{%
\usepackage{parskip}
}{% else
\setlength{\parindent}{0pt}
\setlength{\parskip}{6pt plus 2pt minus 1pt}
}
\setlength{\emergencystretch}{3em}  % prevent overfull lines
\providecommand{\tightlist}{%
  \setlength{\itemsep}{0pt}\setlength{\parskip}{0pt}}
\setcounter{secnumdepth}{0}
% Redefines (sub)paragraphs to behave more like sections
\ifx\paragraph\undefined\else
\let\oldparagraph\paragraph
\renewcommand{\paragraph}[1]{\oldparagraph{#1}\mbox{}}
\fi
\ifx\subparagraph\undefined\else
\let\oldsubparagraph\subparagraph
\renewcommand{\subparagraph}[1]{\oldsubparagraph{#1}\mbox{}}
\fi

%%% Use protect on footnotes to avoid problems with footnotes in titles
\let\rmarkdownfootnote\footnote%
\def\footnote{\protect\rmarkdownfootnote}

%%% Change title format to be more compact
\usepackage{titling}

% Create subtitle command for use in maketitle
\newcommand{\subtitle}[1]{
  \posttitle{
    \begin{center}\large#1\end{center}
    }
}

\setlength{\droptitle}{-2em}

  \title{}
    \pretitle{\vspace{\droptitle}}
  \posttitle{}
    \author{}
    \preauthor{}\postauthor{}
    \date{}
    \predate{}\postdate{}
  

\begin{document}

\subsection{Ecological surveys}\label{ecological-surveys}

\begin{center}\rule{0.5\linewidth}{\linethickness}\end{center}

\paragraph{Survey methods}\label{survey-methods}

Fish surveys were point counts of 7 m radius (Seychelles) or belt
transects of 50 m length (Maldives, Chagos, GBR) conducted on
hard-bottom reef habitat at 8-10 m depth. Surveys were designed to to
minimize diver avoidance or attraction. In point counts, large mobile
species were censused before smaller reef-dwelling species. In belt
transects, large mobile fishes were surveyed in one direction for a 5 m
transect width, and small site-attached species were recorded in the
opposite direction for a 2 m transect width. For both survey types, all
diurnal, non-crypic (\textgreater{}8cm) reef-associated fish were
counted and sized to the nearest centimetre (total length, TL). TL
measurements were calibrated by estimating the length of sections of PVC
pipe and comparing it to their known length prior to data collection
each day. All fish sizes (total length, cm) were then converted to body
mass (grams) using published length \textasciitilde{} weight
relationships (Froese \& Pauly 2017), and standardized by survey area to
give species-level biomass estimates that were comparable across
datasets (kg hectare\textsuperscript{-1}).

Following fish surveys, benthic habitat composition was surveyed with 50
m line transects. Benthos composition was recorded by noting the benthic
taxa directly under the transect line at 50 cm intervals (line intercept
method, REF). Taxa were grouped into broad functional groups (e.g.~CCA,
macroalgae, turf algae) and, if they were hard corals, identified to
genus level. The structural complexity of the reef was visually
estimated on a six-point scale, ranging from 0 (no vertical relief) to 5
(complex habitat with caves and overhangs) (Polunin \& Roberts 1993).

Observation error and bias were minimized because one observer (NAJG)
performed all surveys, except for benthic surveys in Seychelles (SW).
Point counts and belt transects give comparable biomass estimates (REF).

\paragraph{Region details}\label{region-details}

In Seychelles, 21 reefs were surveyed in 2008, 2014, and 2017. Sites
were located on two inhabited islands (Mahe, Praslin) and were
stratified to include carbonate fringing reefs, granitic rocky reefs
with coral growth, and patch reef habitats on a sand, rubble, or rock
base (Fig MAP). Counts were replicated at each site, with 16 surveys
conducted in 2008 and 8 surveys in 2011, 2014 and 2017. Power analysis
indicated that only 8 replicates were needed (REF), and thus we only
considered surveys from 8 replicates per site per survey year. Replicate
survey points were haphazardly located on the reef slope at least 15m
away from each other. Overall, the surveys covered up to 0.5km of reef
front and 2,500m\textsuperscript{2} of reef habitat.

In the GBR, surveys were conducted between November 2010 and January
2011. Five mid-shelf reefs near the city of Townsville were surveyed by
splitting them into 3 wave exposed and 3 wave sheltered sites. Each of
these 6 sites were split up again along a zonal gradient into 3 zones -
crest (2-3m depth), slope (7-9m depth), and flat (100m distance from
crest). Each zone was surveyed with four transects, thus giving 72
transects per reef (four replicates for three wave zones nested in 6
sites per reef) and 360 total transects.

In Chagos, surveys were conducted in 2010 at 25 sites on 4 atolls (Fig
MAP). Belt transects were placed at yadda yadda yadda. Four replicates
transects conducted at each site.

In Maldives, surveys were conducted in 2013 at 11 sites on 1 atoll (Fig
MAP).

\paragraph{Herbivore feeding
observations}\label{herbivore-feeding-observations}

Feeding observations of Indo-Pacific herbivores provided species-level
estimates on bite rates and, for scrapers, bite volumes. Surveys were
conducted in the Red Sea (AH), Indonesia (AH), and GBR (AH and AGL). We
only analysed feeding observations for species observed in the UVC
dataset (39). For each observed fish, we estimated the average feeding
rate (bites per minute). For scrapers, we also estimated the average
bite scar size.

\paragraph{Ecological variable
processing}\label{ecological-variable-processing}

Herbivore species were categorised as grazers, scrapers or browsers
according to published diet observations (REF). Fish biomass estimates
were averaged across replicates at each site to give the total biomass
(kg ha\textsuperscript{-1}) of each functional feeding group (grazers,
scrapers, browsers), and these estimates formed the basis of all
subsequent analyses.

Explanatory covariates were derived from fish and benthic surveys.
First, to account for fishing effects ranging from pristine Chagos reefs
to heavily-exploited Seychelles reefs, we estimated total community
biomass as a proxy for exploitation pressure. Fishable biomass
JUSTIFICATION (REF). Second, benthic surveys provided site-level
estimates of benthic composition. We estimated the site-level cover for
four major habitat-forming groups (live hard coral, macroalgae, sand,
and rubblerubble), and structural complexity, by averaging across
replicates at each site. To understand the range of benthic habitat
types across the dataset, we conducted a PCA to identify common habitat
groupings (i.e.~benthic regimes) (Jouffray et al. 2015). Prior to
statistical modelling, we scaled and centered all continuous covariates
to a mean of zero and standard deviation of one, and converted the
categorical protection covariate to two dummy variables (fished -
protected, fished - pristine) (Schielzeth 2010).

\subsection{Statistical analyses}\label{statistical-analyses}

\begin{center}\rule{0.5\linewidth}{\linethickness}\end{center}

\paragraph{Biomass analysis}\label{biomass-analysis}

We used linear mixed effects to model to examine variation in herbivore
biomass along gradients in benthic habitat composition and fishing
intensity. Benthic fixed effects were hard coral, macroalgae, sand,
rubble and structural complexity, and fishing fixed effects were
fishable biomass and fishing protection. Models were fitted separately
for each functional feeding group for normally distributed errors, and
biomass was log\_10\_ transformed. Parameter estimates were extracted to
compare effect sizes of each explanatory covariate, and fitted
relationships were visualized with model predictions that excluded
random effects.

\paragraph{Grazing function analysis}\label{grazing-function-analysis}

Fish functions were defined separately for each functional feeding
group, and feeding observations used to convert UVC observations to one
of three grazing functions. We used a Bayesian hierarchical modelling
framework that estimate species- and genera-level functional rates,
which allowed us to estimate grazing for UVC species which were not
observed in feeding surveys (63). For grazers, which feed on turf and
other filamentous algal material, herbivory maintains turf in cropped
states that are expected to be benign states for coral recruitment.
Grazing function was quantified in terms of potential feeding intensity,
measured as the total number of bites per hour. Feeding observations
were modelled with species- and genera-specific bite rates:

\(bite_rate = Gamma(\mu, \theta)\)
\(log(\mu) = X + species_i + genus_j\)

Grazing bite rates were unrelated to body size (Fig. SX), and thus we
did not consider potential size differences in functional rates.

For scrapers and excavators, `scraping' of endolithic and detrital
material exposes substrate and promotes coral recruitment. Scraping
function was quantified in terms of area of substrate removed per hour.
Feeding observations provided estimates of bite rates and average scar
sizes (scar\_area), which we modelled with species- and genera-specific
grazing rates.

\(bite = Gamma(\mu, \theta)\)
\(log(\mu) = X + B*TL + species_i + genus_j\)

\(scar_area = Gamma(\mu, \theta)\)
\(log(\mu) = X + B*TL + species_i + genus_j\)

\(area_removed = Normal(\mu, \sigma)\) \(\mu = X + ??\)

By including size (TL) as an explanatory covariate, our model accounted
for body size effects, as scar sizes increased with body size (Fig. SX)
and bite rates decreased with body size (Fig. SX).

For browsers, consumption of established macroalgae acts to suppress
macroalgal growth (REF), though grazing can be inhibited at hig
macroalgal densities (Hoey \& Bellwood 2011). Browsing function was
quantified in terms of algal material removed per hour. Feeding
observations provided estimates of bite rates and ????. Browsing feeding
observations were limited to two species from two genera, so we only
fitted genera-specific random effects:

\(bite = Normal(\mu, \sigma)\)

\(\mu = X + genus_j\)


\end{document}
